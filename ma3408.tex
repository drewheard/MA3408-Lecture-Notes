\documentclass[a4paper]{tufte-book}
\usepackage{subfiles} 
\usepackage{microtype,mathrsfs}
\usepackage{amsmath,amsthm}
\usepackage[all]{xypic}
\usepackage{pgf,tikz}
\usepackage{graphicx}
\usepackage{cleveref}
\hypersetup{colorlinks}
\usepackage{quiver,mathtools}
\usetikzlibrary{arrows}
\usetikzlibrary{matrix}
%\geometry{showframe}% for debugging purposes -- displays the margins
\usepackage{tikz-cd}
\setcounter{secnumdepth}{2}
\title{MA3408 - Algebraic Topology II}
\date{\today}
\author{Drew Heard}
\usepackage{amsmath}
\usepackage[shortlabels]{enumitem}
\usepackage{fontawesome}
\usepackage[most]{tcolorbox}
    \tcbuselibrary{theorems}
    \newtcbtheorem{exercise}{Exercise}{}{exercise}
%------------------------------------------------------------------------------
%----- Commands and macros ----------------------------------------------------
%------------------------------------------------------------------------------

%----- Thms, Defs, etc. -------------------------------------------------------
\numberwithin{equation}{section}
\swapnumbers %pour que le numero apparaisse devant le theoreme

\newtheorem{Thm}[equation]{Theorem}
%\newtheorem{MainThm}[equation]{Main Theorem}
\newtheorem*{Thm*}{Theorem}
%\newtheorem*{MainThm*}{Main Theorem}
%\newtheorem{ThmDef}[equation]{Theorem and Definition}
\newtheorem{Prop}[equation]{Proposition}
%\newtheorem{PropDef}[equation]{Proposition and Definition}
\newtheorem{Lem}[equation]{Lemma}
%\newtheorem{MainLem}[equation]{Main Lemma}
\newtheorem{Cor}[equation]{Corollary}
\newtheorem{CorConj}[equation]{Corollary of Conjecture}
%\newtheorem*{Cor*}{Corollary}
\newtheorem{Conj}[equation]{Conjecture}
\newtheorem{Leap}[equation]{Leap of Faith}
%\newtheorem{Axiom}[equation]{Axiom}

%\theoremstyle{definition}
\theoremstyle{remark}
\newtheorem{Def}[equation]{Definition}
\newtheorem{Ter}[equation]{Terminology}
%\newtheorem{Defs}[equation]{Definitions}
\newtheorem{Not}[equation]{Notation}
%\newtheorem{Nots}[equation]{Notations}
%\newtheorem{DefNot}[equation]{Definition-Notation}
\newtheorem{Ex}[equation]{Exercise}
\newtheorem{Exas}[equation]{Examples}
\newtheorem{Exa}[equation]{Example}
\newtheorem{Cons}[equation]{Construction
}\newtheorem{Conv}[equation]{Convention}
\newtheorem{Hyp}[equation]{Hypothesis}

%\theoremstyle{remark}
\newtheorem{Rem}[equation]{Remark}
%\newtheorem{Rems}[equation]{Remarks}
\newtheorem{Que}[equation]{Question}
%\newtheorem{Baksh}[equation]{Baksheesh}
\newcommand{\dmo}{\DeclareMathOperator}
\newcommand{\xr}{\xrightarrow}
\Crefname{tcb@cnt@exercise}{Exercise}{Exercise 1}
\Crefname{Exa}{Example}{Examples}
\newcommand{\bbZ}{\mathbb{Z}}
\dmo{\Hom}{Hom}
\dmo{\coker}{coker}
\dmo{\im}{im}
\dmo{\colim}{colim}
\dmo{\Ext}{Ext}
\dmo{\Sp}{Sp}
\dmo{\Tor}{Tor}
\dmo{\Set}{Set}
\dmo{\Ab}{Ab}
\dmo{\Top}{Top}
\dmo{\Ring}{Ring}
\dmo{\op}{op}
\setcounter{tocdepth}{2}
\setcounter{secnumdepth}{2}
\newcommand{\cat}[1]{\mathcal{#1}}%or: 
\begin{document}
\maketitle
\tableofcontents
\subfile{section1}
\subfile{section2}
\appendix
\chapter{A nice category of topological spaces}
\section{The compact open topology}
In this appendix we briefly discuss how to give the set of continuous maps between topological spaces $X$ and $Y$ a topology, such that the product is left adjoint to the Hom functor. To begin, we fix some notation. 
\begin{Rem}
Let $X$ and $Y$ be topological spaces. Let $M(X,Y)$ denote the \emph{set} of continuous homomorphisms from $X$ to $Y$. There is an evaluation map 
\[
e' \colon \Hom_{\mathop{Sets}}(X,Y) \times X \to Y
\]
given by $e'(f,x) = f(x)$.  This restricts to a function
\[
e \colon M(X,Y) \times X \to Y.
\]
\end{Rem}
\begin{Def}
A topology on $M(X,Y)$ is called admissible if $e$ is continuous with respect to this topology. 
\end{Def}
\begin{Rem}
It is possible that $M(X,Y)$ has no admissible topologies. 
\end{Rem}
\begin{Def}
The compact-open topology on $M(X,Y)$ has as a sub-base the family of sets 
\[
U^K = \{f \in M(X,Y)\mid f(K) \subseteq U\}
\]
where $K \subseteq U$ is compact and $U$ is open in $Y$. 
\end{Def}
\begin{Prop}
If $X$ is a locally compact\footnote{i.e., every point in $X$ has a compact neighborhood} Hausdorff space, the the compact-open topology on $M(X,Y)$ is admissible. 
\end{Prop}
\begin{Rem}
The compact-open topology is the coarsest admissible topology: for any admissible topology $\tau$ we have $\tau_{\text{co}} \subseteq \tau$. 
\end{Rem}
\begin{Rem}
Suppose we have sets $X,Y,X$. Then there is an adjoint equivalence
% https://q.uiver.app/?q=WzAsMixbMCwwLCJcXEhvbV97XFxtYXRob3B7U2V0c319KFggXFx0aW1lcyBZLFopICJdLFsxLDAsIlxcSG9tX3tcXG1hdGhvcHtTZXRzfX0oWCxcXEhvbV97XFxtYXRob3B7U2V0c319KShZLFopIl0sWzAsMSwiXFxwaGkiLDAseyJvZmZzZXQiOi0xfV0sWzEsMCwiXFxwc2kiLDAseyJvZmZzZXQiOi0xfV1d
\[\begin{tikzcd}
    {\Hom_{\mathop{Sets}}(X \times Y,Z) } & {\Hom_{\mathop{Sets}}(X,\Hom_{\mathop{Sets}})(Y,Z)}
    \arrow["\phi", "\sim"', shift left=1, from=1-1, to=1-2]
    \arrow["\psi", shift left=1, from=1-2, to=1-1]
\end{tikzcd}\]
given by
\[
\phi(f)(x)(y) = f(x,y) \quad \text{ and } \quad \psi(g)(x,y) = g(x)(y).
\]
\end{Rem}
\begin{Prop}
If $X,Y,X$ are topological spaces with $Y$ Hausdorff, locally compact, then 
\[
\phi \colon M(X \times Y,Z) \xrightarrow{\cong} M(X,M(Y,Z))
\]
is an isomorphism of sets. If $X$ is Hausdorff, then it is a homeomorphism (using the compact-open topology). 
\end{Prop}
\end{document}