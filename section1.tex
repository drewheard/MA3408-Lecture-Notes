\documentclass[ma3408.tex]{subfiles}
\begin{document}
\chapter{Homotopy theory}
\section{Review of basics on homotopy theory}
We begin with a recollection of some facts that have been covered in Algebraic Topology I and Introduction to Topology. 
\begin{Not}
We let $I = [0,1]$ denote the unit interval. For a pointed topological space $X$ we will denote the basepoint by $x_0$ or $\ast$. 
\end{Not}
We recall the following definition. 
\begin{Def}
	A homotopy between $f,g \colon X \to Y$ is a continuous function $H \colon X \times I \to Y$ such that $H(x,0) = f(x)$ and $H(x,1) = g(x)$ and $H(x_0,t) = y_0$ for all $t \in I$. We will write $f \simeq g$, or $f \simeq_H g$, if we need to make the choice of homotopy clear. \begin{marginfigure} \centering\includegraphics[scale = 0.5]{path_homotopy}\caption{A homotopy between $f$ and $g$.}\label{fig:homotopy}\end{marginfigure}

	For a subspace $A \subseteq X$, a relative homotopy is a homotopy with $H(a,t) = f(a) = g(a)$ for all $a \in A, t \in I$. 
	\end{Def}

\begin{Rem}
	Equivalently, we can specify a family of continuous maps $h_t \colon X \to Y$ such that $h_0 = f , h_1 = g$ and
\[
\begin{split}
H &\colon X \times I \to Y\\
 (x,t) &\mapsto h_t(x)
\end{split}
\]
is continuous. We will switch between the two equivalent definitions without comment, using whatever is more convenient. 
\end{Rem}
\begin{Prop}
For all spaces $X$ and $Y$, homotopy is an equivalence relation on the set of maps from $X$ to $Y$. Furthermore, if we are given $k \colon A \to X,\ell \colon Y \to B$ and homotopic maps $f \simeq g \colon X \to Y$, then $fk \simeq gk \colon A \to Y$ and $\ell f \simeq \ell g \colon X \to B$. 
\end{Prop}
\begin{proof}
Let $f,g \colon X \to Y$, then
\begin{enumerate}
	\item $f \simeq_F f$ via $F(x,t) = f(x)$ for all $x \in X,t \in I$. 
	\item If $f \simeq_F g$, then $g \simeq_G f$ where $G(x,t) = F(x,1-t)$. 
	\item If $f \simeq_F g$ and $g \simeq_G h$, then $f \simeq_H h$ via
	\[
H(x,t) = \begin{cases}
F(x,2t) & \text{ if } 0 \le t \le 1/2 \\
G(x,2t-1) & \text{ if } 1/2 \le t \le 1. 
\end{cases}
	\]
\end{enumerate}
For the last part of the proposition let $f_t$ be a homotopy between $f$ and $g$, then $f_tk$ and $\ell f_t$ give the required homotopy. 
\end{proof}
\begin{Def}
For a map $f \colon X \to Y$, we let $[f]$ denote the equivalence class containing $f$. The collection of all homotopy classes of maps from $X$ to $Y$ is denoted $[X,Y]$.\sidenote{If our spaces are based, then these should be homotopy classes of \emph{based} maps.} 
\end{Def}
\begin{Rem}
Note that if $\alpha = [f] \in [Y,Z]$ and $\beta = [g] \in [X,Y]$, then $\alpha\beta = [f \circ g] \in [X,Z]$, i.e., we can form the category $hTop_*$ whose objects are topological spaces, and whose morphisms are homotopy classes of maps. 
\end{Rem}
\begin{Rem}
We now very quickly review a number of standard topological constructions. 
\begin{itemize}
	\item Let $X$ be a space and $A \subseteq X$. A map $r \colon X \to A$ such that $ri(a) = a$ for all $a \in A$ is called a retraction of $X$ onto $A$, and $A$ is called a retract of $X$. 
	\item Let $i \colon A \hookrightarrow X$ be the inclusion, so that $ri = \text{id}_A$. If $ir \simeq \text{id}_X$, we call this a deformation retraction, and say that $A$ is a deformation retract of $X$. 
	\item If $f \colon X \to Y$, then a section of $f$ is a map $s \colon Y \to X$ such that $f \circ s = \text{id}_Y$. We can also ask for a \emph{homotopy} section by requiring only that $f \circ s \simeq \text{id}_Y$. 
\end{itemize}
\end{Rem}
\begin{Def}
A map $f \colon X \to Y$ is called null-homotopic if $f \colon c_y \colon X \to Y$ where $c_y X \to Y$ is the constant map sending all of $X$ to the point $y \in Y$. A homotopy between $f$ and $c_y$ is called a null-homotopy. A space $X$ is contractible if $\text{id}_X$ is null-homotopic. 
\end{Def}
\begin{Def}
Let $(X,x_0)$ be a based topological space and $X \times I$ the cylinder on $X$. The quotient
\[
CX = (X \times I)/(X \times \{ 1 \} \cup \{ x_0 \} \times I)
\]
with the base-point the equivalence class of $(x_0,1)$ is called the (reduced) cone on $X$. Note that we have a natural inclusion $X \to CX$ of based maps given by $x \mapsto [x,0]$. 
\end{Def}
\begin{Lem}\label{lem:cone_is_contractible}
The cone $CX$ is contractible. 
\end{Lem}
\begin{proof}
Define $F \colon CX \times I \to CX$ by
\[
F([x,t],s) = [x,s+(1-s)t]. 
\]
Note then that we have
\[
F([x,t],0) = [x,t] \quad \text{ and } \quad F([x,t],1) = [x,1]. \qedhere
\]

\end{proof}
\begin{Lem}
The following are equivalent:
\begin{enumerate}[label=(\roman*)]
	\item $f \colon X \to Y$ is null-homotopic. 
	\item $f$ can be extended to $CX$:
	% https://q.uiver.app/?q=WzAsMyxbMCwwLCJYIl0sWzEsMCwiWSJdLFswLDEsIkNYIl0sWzAsMSwiZiJdLFswLDIsImkiLDIseyJzdHlsZSI6eyJ0YWlsIjp7Im5hbWUiOiJob29rIiwic2lkZSI6InRvcCJ9fX1dLFsyLDEsIlxcZXhpc3RzIFxcdGlsZGUgZiIsMix7InN0eWxlIjp7ImJvZHkiOnsibmFtZSI6ImRhc2hlZCJ9fX1dXQ==
\[\begin{tikzcd}
	X & Y \\
	CX
	\arrow["f", from=1-1, to=1-2]
	\arrow["i"', hook, from=1-1, to=2-1]
	\arrow["{\exists \tilde f}"', dashed, from=2-1, to=1-2]
\end{tikzcd}\]
\end{enumerate}
\end{Lem}
\begin{proof}
$(i) \implies (ii): $ Suppose $f$ is null-homotopic, so $f \simeq_F \ast$. Then $F(X\times \{ 1 \} \cup \{ \ast \} \times I) = \ast$, so by the universal property of the quotient, we can find $\tilde F \colon CX \to Y$ such that $\tilde f \circ i = f$. 

$(ii) \implies (i): $ Suppose $\tilde f \circ i = f$, then because $CX$ is contractible (\Cref{lem:cone_is_contractible}), we have $f = \tilde f \circ \text{id}_{CX} \circ i \simeq \tilde f \circ (\ast_{CX}) \circ i \simeq \ast$, so that $f$ is null-homotopic. 
\end{proof}
\begin{Def}
A map $f \colon X \to Y$ is a homotopy equivalence if there exists $g \colon Y \to X$ such that $fg \simeq \text{id}_Y$ and $gf \simeq \text{id}_X$. We write $X \simeq Y$. 
\end{Def}
\begin{Exa}
\begin{enumerate}[label=(\roman*)]
	\item $X$ is contractible if and only if $X \simeq \ast$. 
	\item If $i \colon A \hookrightarrow X$, and $r \colon X \to A$ is a deformation retract, then $i$ and $r$ are homotopy equivalences, and $A \simeq X$. 
\end{enumerate}
\end{Exa}
\section{Higher homotopy groups}

\begin{Not} We will let $I_n = I^{\times n}, \partial I^n$ be the boundary of $I^n$, and write $[-,-]$ for homotopy classes of maps (if our spaces are based, these fix the base point).
\end{Not}
\begin{Def}
	For each $n \ge 0$ and $X$ a topological space with $x_0 \in X$, we define
	\[
\pi_n(X, x_0) = [(I^n, \partial I^n), (X, x_0)].
\]
\end{Def}
\begin{Rem}
	\begin{enumerate}[(i)]
		\item When $n = 0$, we have $I^0 = \text{pt}$ and $\partial I^0 = \emptyset$, therefore $\pi_0(X)$ is the set of path components of $X$.
		\item When $n = 1$, this is a group, but need not be abelian (for example, consider the wedge of two circles).
		\item Note that $I^n / \partial I^n \simeq S^n$ and $\partial I^n / \partial I^n \simeq s_0$. By the universal property of the quotient map, we see that 
		\[
		\pi_n(X, x_0) \cong [(S_n, s_0), (X, x_0)].
\]
	\end{enumerate}
\end{Rem}
\begin{Def}
	A maps of pairs $(X,A) \to (Y,B)$ is a map $f \colon X \to Y$ with $f(A) \subseteq B$, i.e., the diagram:
% https://q.uiver.app/?q=WzAsNCxbMCwwLCJBIl0sWzEsMCwiQiJdLFswLDEsIlgiXSxbMSwxLCJZIl0sWzAsMV0sWzIsMywiZiIsMl0sWzAsMiwiIiwxLHsic3R5bGUiOnsidGFpbCI6eyJuYW1lIjoiaG9vayIsInNpZGUiOiJib3R0b20ifX19XSxbMSwzLCIiLDEseyJzdHlsZSI6eyJ0YWlsIjp7Im5hbWUiOiJob29rIiwic2lkZSI6ImJvdHRvbSJ9fX1dXQ==
\[\begin{tikzcd}
	A & B \\
	X & Y
	\arrow[from=1-1, to=1-2]
	\arrow["f"', from=2-1, to=2-2]
	\arrow[hook', from=1-1, to=2-1]
	\arrow[hook', from=1-2, to=2-2]
\end{tikzcd}\]
	commutes. 
\end{Def}
\begin{Prop}
	If $n \ge 1$, then $\pi_n(X,x_0)$ is a group with respect to the operation
	\[
(f+g)(t_1,\ldots,t_n) = \begin{cases}
	f(2t_1,t_2,\ldots,t_n) & 0 \le t_1 \le 1/2 \\
	g(2t_1-1,t_2,\ldots,t_n) & 1/2 \le t_1 \le 1. 
\end{cases}
	\]
\end{Prop}
\begin{proof}
	The identity is given by the constant map taking all of $I^n$ to $x_0$ and the inverse of $f$ is given by 
	\[
-f(t_1,\ldots,t_n) = f(1-t_1,t_2,\ldots,t_n). \qedhere
	\]
\end{proof}
\begin{Rem}
	Call the group operation $+_1$. Note that we can also define an operation $+_i$ for $1 \le i \le n$ by the same formula on the $i$-th coordinate. 
\end{Rem}
\begin{Thm}
	All of these operations agree, and for $n \ge 2$, these give $\pi_n(X,x_0)$ the structure of an abelian group. 
\end{Thm}
This is a consequence of the following exercise, known as the Eckmann--Hilton lemma. 
\begin{exercise}{Eckmann--Hilton lemma}{}
	Let $M$ be a set and let $\ast,\bullet$ be two binary operations on $M$, $\ast,\bullet \colon M \times M \to M$, both with unit elements. Suppose that 
	\[
(a \ast b) \bullet (c \ast d) = (a \bullet c) \ast (b \bullet d)
	\]
	for all $a,b,c,d \in M$. Show that the units agree, these two operations agree, and that the multiplication is commutative and associative. 
\end{exercise}
\begin{Rem}
	Let use show that 
	\[
(f+_1 g) +_2 (h+_1 i) \simeq (f+_2 h) +_1 (g+_2 i). 
	\]
	Indeed, both of these are the following map
	\[
(t_1,t_2,\ldots,) \mapsto \begin{cases}
	f(2t_1,2t_2,\ldots,) &[1/2,0] \times [1/2,0]\\
	g(2t_1-2,2t_2,\ldots,) & [1/2,1] \times [0,1/2]\\
h(2t_1,2t_2-2,\ldots) & [0,1/2] \times [1/2,1]\\
i(2t_1-1,2t_2-2,\ldots) & [1/2,1] \times [1/2,1].
\end{cases}
	\]
\end{Rem}
\begin{Rem}
	Another approach is given by the following visualization: 
\begin{figure}[h!] \centering\includegraphics[scale = 0.5]{abelian.png}\caption{$f + g \simeq g +f $.}\label{fig:abelian}\end{figure}
	That is, so long as $n \ge 2$, we can shrink the domain of $f$ and $g$ to smaller cubes (mapping the remaining region to the base point), slide $f$ and $g$ past each other, and then increase the domains back again. 
\end{Rem}
\begin{exercise}{}{}
	Let $G$ be a topological group with identity element $e$, then $\pi_1(G,e)$ is abelian. \\
	\textbf{Hint: } Use Eckmann--Hilton, or note the following: A topological group is a group object in the category of topological spaces. What is a group object in the category of groups? 
\end{exercise}
\begin{Prop}
	If $n \ge 1$ and $X$ is path connected, then there is an isomorphism $\beta_{\gamma} : \pi_n(X, x_0) \xrightarrow{\simeq} \pi_n(X, x_0)$ given by $\beta_{\gamma}([ f ]) = [\gamma \circ f ]$ where $\gamma$ is a path in $X$ from $x_1$ to $x_0$ and $\gamma \circ f$ is constructed by first shrinking the domain of $f$ to a smaller cube inside of $I^n$, and then inserting the path $\gamma$ radially from $x_1$ to $x_0$ on the boundaries of these cubes.
	\begin{figure}[h] \centering\includegraphics[scale = 0.5]{path.png}\caption{$\beta_{\gamma}$.}\label{fig:path}\end{figure}
\end{Prop}
\begin{proof}
	Observe the following:
	\begin{enumerate}
		\item $\gamma \circ (f + g) \simeq \gamma \circ f + \gamma \circ g$, i.e., $\beta_{\gamma}$ is a group homomorphism.
		\item $(\gamma \circ \eta) \circ f \simeq \gamma \circ (\eta \circ f)$, for $\eta$ a path from $x_0$ to $x_1$. 
		\item $c_{x_0} \circ f \simeq f$, where $c_{x_0}$ denotes the constant path based at $x_0$. 
		\item $\beta_{\gamma}$ is well-defined with respect to homotopies of $f$ or $\gamma$. 
	\end{enumerate}

	The only point that is perhaps not clear is (i). For this, we deform $f$ and $g$ to be constant on the right and left halves of $I^n$, respectively, producing maps we call $f+0$ and $0 + g$. We then excise a wider symmetric middle slab of $\gamma(f+0)$  and $\gamma(0+g)$ until it becomes $\gamma(f+g)$:
\begin{figure}[h] \centering\includegraphics[scale = 0.5]{path_invariance.png}\end{figure}
\end{proof}
\begin{Rem}
	Therefore if $X$ is path-connected, different choices of base point $x_0$ yield isomorphic groups $\pi_n(X,x_0)$, which may then simply be written as $\pi_n(X)$. 
\end{Rem}
\begin{Lem}
	If $\{ X_{\alpha} \}$ is a collection of path-connected spaces, then $\pi_n(\prod_{\alpha} X_{\alpha}) \cong \prod_{\alpha} \pi_n(X_{\alpha})$. 
\end{Lem}
\begin{proof}
	Note that $\Hom(Y,\prod_{\alpha} X_{\alpha}) \simeq \prod_{\alpha} \Hom(Y,X_{\alpha})$. In particular, a map $S^n \to \Hom(Y,\prod_{\alpha}X_{\alpha})$ is determined by a collection of maps $S^n \to X_{\alpha}$. Likewise, a homotopy $S^n \times I \to \prod_{\alpha} X_{\alpha}$ is determined by a colletion of homotopies $S^n \times I \to X_{\alpha}$. This implies the result. 
\end{proof}
\begin{Prop}
	Homotopy groups are functorial: given a map $\phi \colon X \to Y$ we get group homomorphisms $\phi_{\ast} \colon \pi_n(X,x_0) \to \pi_n(X,\phi(x_0))$ given by $[f] \mapsto [\phi \circ f]$ for all $n \ge 1$. 
\end{Prop}
\begin{proof}
We have the following:
	\begin{enumerate}
	\item $\phi_*$ is well-defined: if $f \simeq g$ via $\psi_t$, then $\phi \circ \psi_t$ defines a homotopy between $\phi \circ f$ and $\phi \circ g$. 
	\item This is a group homomorphism: $\phi \circ (f+g) \simeq \phi \circ g + \phi \circ g$ by the definition of the addition operation. Therefore. 
	\[
\phi_*[f+g] = \phi_*[f] + \phi_*[g]. 
	\]
\end{enumerate}
\end{proof}
\begin{exercise}{}{}
	If $\phi \colon X \to Y$ is homotopy equivalence (not necessarily base-point preserving), then $\pi_* \colon \pi_n(X,x_0) \to \pi_n(Y,\phi(y_0))$ is an isomorphism.  
\end{exercise}
\begin{Rem}
	We recall the following lifting property: Suppose $p \colon (\tilde X,\tilde x_0) \to (X,x_0)$ is a covering, and there is a map $f \colon (Y,y_0) \to (X,x_0)$ with $Y$ path-connected and locally path-connected. Then a lift $\tilde f$ exists if and only if $f_*\pi_1(Y,y_0) \subseteq p_*\pi_1(\tilde X,\tilde x_0)$. 
	% https://q.uiver.app/?q=WzAsMyxbMCwxLCIoWSxZXzApIl0sWzEsMSwiKFgseF8wKSJdLFsxLDAsIihcXHRpbGRlIFgseF8wKSJdLFswLDEsImYiXSxbMiwxXSxbMCwyLCJcXHRpbGRlIGYiLDAseyJzdHlsZSI6eyJib2R5Ijp7Im5hbWUiOiJkYXNoZWQifX19XV0=
\[\begin{tikzcd}
	& {(\tilde X,\tilde x_0)} \\
	{(Y,y_0)} & {(X,x_0)}
	\arrow["f", from=2-1, to=2-2]
	\arrow["p",from=1-2, to=2-2]
	\arrow["{\tilde f}", dashed, from=2-1, to=1-2]
\end{tikzcd}\]

\end{Rem}
\begin{Prop}
	If $p$ is a covering, then $p_* \colon \pi_n(\tilde X,\tilde x_0) \to \pi_n(X,x_0)$ is an isomorphism for all $ n \ge 2$. 
\end{Prop}
\begin{proof}
Let us first show surjectivity. To that end, suppose we have a map $f \colon (S^n,s_0) \to (X,x_0)$ where $n \ge 2$. The assumption on  $n$ gives $\pi_1(S^n) = 0$, so $f_*\pi_1(S^n,s_0) \subseteq p_*\pi_1(\tilde X,\tilde x_0)$ holds. We therefore find a lift in the following:
\[\begin{tikzcd}
	& {(\tilde X,\tilde x_0)} \\
	{(S^n,s_0)} & {(X,x_0)}
	\arrow["f", from=2-1, to=2-2]
	\arrow["p", from=1-2, to=2-2]
	\arrow["{\tilde f}", dashed, from=2-1, to=1-2]
\end{tikzcd}\]
Then $p_*[\tilde f] = [f]$, and $p_*$ is surjective. 

To see that $p_*$ is injective, let $[\tilde f] \in \ker(p_*)$, i.e., $p_*[\tilde f] = [p \circ \tilde f] = 0$. Let $f = p \circ \tilde f$, then this is homotopic to the constant map $f \simeq c_{x_0}$ via a homotopy $\phi_t \colon (S^n,s_0) \to (X,x_0)$ with $\phi_1 =f $ and $\phi_0 = c_{x_0}$. By the same argument as above, the homotopy $\phi_t$ can be lifted to $\tilde \phi_t$. This satisfies $p \circ \tilde \phi_1 \simeq \phi_1 $ and $p \circ \tilde \phi_0 \simeq \phi_0$. By the uniqueness of lifts, we must have $\tilde \phi_1 \simeq \tilde f$ and $\tilde \phi_0 \simeq c_{x_0}$. In other words, $\tilde \phi_t$ gives a homotopy between $\tilde f$ and $c_{x_0}$, so that $[\tilde f]  = 0$, and $p_*$ is injective. 
\end{proof}
\begin{Exa}
	$S^1$ has universal cover $p \colon \mathbb{R} \to S^1$, $p(t) = e^{2\pi i t}$. Then $\pi_n(S^1) \cong \pi_n(\mathbb{R}) \cong 0$ for $n \ge 2$. 
\end{Exa}
\begin{exercise}{}{}
	Find two spaces $X,Y$ with $\pi_nX \cong \pi_nY$ but $X \not \simeq Y$. 

	\textbf{Hint: } What is the universal cover of $\mathbb{R}P^n$?
\end{exercise}
\begin{Rem}[Relative homotopy groups]
	Suppose we have $(X,x_0)$ and a subspace $A$ containing $x_0$. We note that $i_* \colon \pi_n(A,x_0) \to \pi_n(X,x_0)$ is not injective in general (example, take $S^1$ into $\mathbb{R}^2$). An element in the kernel of $i_*$ is a map $f \colon (I^n,\partial I^n) \to (A,x_0)$ such that 
	\[
(I^n,\partial I^n) \xrightarrow{f} (A,x_0) \xrightarrow{i} (X,x_0)
	\] 
	is homotopic to $c_{x_0}$. This means there exists a homotopy
	\[
H \colon I^n \times I \to X
	\]
	such that $H(-,1) = f, H(-,0) = c_{x_0}$ and $H \mid_{\partial I^n \times I} = c_{x_0}$. 

	If we define $J^n = I^n \times \{ 0 \} \cup \partial I^n \times I \subseteq I^n \times I$, then this is a map of triples
	\[
H \colon (I^{n+1},\partial I^{n+1},J^n) \to (X,A,x_0). 
	\] 
\end{Rem}
\begin{Def}
	\[
\pi_n(X,A,x_0) = [(I^n,\partial I^n,J^{n-1}),(X,A,x_0)]
	\]
\end{Def}
\begin{Rem}
	Equivalently, 
	\[
\pi_n(X,A,x_0) = [(D^n,S^{n-1},s_0),(X,A,x_0)]
	\]
\end{Rem}
\begin{Prop}
	If $n \ge 2$, then $\pi_n(X,A,x_0)$ is a group, and if $n \ge 3$, then it is abelian. 

	For all $n \ge 2$, a map of pairs $\phi \colon (X,A,x_0) \to (Y,B,y_0)$ induces homomorphisms $\phi_* \colon \pi_n(X,A,x_0) \to \pi_n(Y,B,y_0)$ for all $n \ge 2$. 
\end{Prop}
\begin{proof}
This is similar to the case of $\pi_n(X)$ itself, and the details are left to the reader. 
\end{proof}
\begin{Thm}\label{thm:les_rel}
	The relative homotopy groups $(X,A,x_0)$ fit into a long exact sequence 
	\[
\cdots \to \pi_n(A,x_0) \xr{i_*} \pi_n(X,x_0) \xr{j_*} \pi_n(X,A,x_0) \xr{\partial_n} \pi_{n-1}(A,x_0) \to \cdots
	\]
	where the map $\partial_n$ is defined by $\partial_n([f]) = [f\mid_{I^{n-1}}]$. 
\end{Thm}
The proof relies on the following.
\begin{Lem}[Compression criterion]
	A map $f \colon (D^n,S^{n-1},x_0) \to (X,A,x_0)$ represents 0 in $\pi_n(X,A,x_0)$ if and only if $f \sim g \text{ rel } S^{n-1}$, where $g$ is a map whose image is contained entirely in $A$. 
\end{Lem}
\begin{proof}
	Suppose $[f] = [g]$ with $g$ as in the statement of the lemma. Note that there is a deformation of $D^n$ onto $x_0$, and so $[f] = [g] = 0$ in $\pi_n(X,A,x_0)$. 

	Conversely, suppose that $[f]$ represents 0 in $\pi_n(X,A,x_0)$. This means there exists a homotopy, relative to $S^{n-1}$, $F \colon D^n \times I \to X$ with $F \mid_{D^n \times \{0 \}} = f$, $F \mid_{D^n \times 1} = c_{x_0}$ and $F \mid_{S^{n-1} \times I} \subseteq A$. We can restrict $F$ to a family of $n$-disks in $D^n \times I$ starting with $D^n \times \{ 0 \}$ and ending with the disk $D^n \times \{1 \} \cup S^{n-1} \times \{ 1\}$, all the disks in the family having the same boundary, then we get a homotopy from $f$ to a map in $A$, stationary on $S^{n-1}$ (said in other words, we can deformation retract $D^n \times [0,1]$ onto $D^n \times \{1 \} \cup S^{n-1} \times I$). 
\end{proof}
We now prove the existence of the long exact sequence.\sidenote{This is the type of proof that is best done by the reader themselves.}
\begin{proof}[Proof of \Cref{thm:les_rel}]
	\textbf{Step 1. }Let us first show exactness at $\pi_n(X,x_0)$. \\

	We first show $\im(i_*) \subseteq \ker(j_*)$. Note that $j_*i_*$ is induced by the composition $j \circ i$ and that these are both inclusion maps. Therefore, for $[f] \in \pi_n(A,x_0)$ we have $j_*i_*[f] = [j \circ i \circ f]$, but this has image contained in $A$, and so $j_*i_*[f] = 0$. This shows $\im(i_*) \subseteq \ker(j_*)$. 

	To see the converse (namely, $\ker(j_*) \subseteq \im(i_*)$) let $[f] \in \ker(j_*)$, i.e. $[j \circ f] = 0$. Note that again $j$ is an inclusion map, and by the compression criteria $f \simeq g'$ relative to $S^{n-1}$, where $g'$ has image contained in $A$. Since $x_0 \in S^{n-1}$, the homotopy fixes the basepoint, i.e, $[f] = [g'] \in \pi_n(X,x_0)$. But because $g'$ has image in $A$, $[g'] \in \pi_n(A,x_0)$ and $i_*[g'] = [i \circ g'] = [f]$, so $[f] \in \im(i_*)$. 

	\textbf{Step 2. } Let us now show exactness at $\pi_n(X,A,x_0)$. \\

	Note that the composite $\partial \circ j_* = 0$ since the restriction of a map $(I^n,\partial I^n,J^{n-1}) \to (X,x_0,x_0)$ to $I^{n-1}$ has image $x_0$ and so represents $0$ in $\pi_{n-1}(A,x_0)$. Therefore, $\im(j_*) \subseteq \ker(\partial)$. For the converse, suppose $[f] \in \ker(\partial)$. This means there exists a basepoint preserving homotopy $H \colon I^{n-1} \times I \to A$ (relative to $\partial I^{n-1}$) from $f \mid_{I^{n-1} \times \{ 0 \}}$ to the constant map where the image of $H$ is contained entirely in $A$. We can then define another homotopy $H$, such that $G_0 = f$, $G_t \mid_{I^{n-1}} = H_t$ and the rest of the image of $G_t$ is $f[I^n]$ union with the image of $H_s$ for $0 \le s \le t$. This homotopy maps $S^{n-1}$ into $A$ at all times, so $[f] = [G_1]$. Moreover, $G_1$ maps the boundary of $I^n$ to $x_0$, so $[G_1] \in \pi_n(X,x_0)$. Altogether, 
	\[
j_*[G_1] = [j \circ G_1] = [G_1] = [f]
	\]
	so $\ker(\partial) \subseteq \im(j_*)$. 


\textbf{Step 3:} Exactness at $\pi_n(A,x_0)$.
	
	Let $[f] \in \pi_n(X,A,x_0)$ then $i_*\partial \in \pi_{n-1}(X,x_0)$ is the class represented by $f \mid_{I^{n-1}}$ and this is homotopic relative $J^{n-2}$ to the constant map to $x_0$, via $f$ viewed as a homotopy. So this implies $\im(\partial_*) \subseteq \ker(i_*)$. Conversely, let $[f] \in \ker(i_*)$ i.e., $i_*[f]= [i \circ f]= 0$. Therefore, there exists a homotopy $H$ between $f$ and a constant map through a homotopy that has image in $X$ and preserves $x_0$. Since $H_0 = f$ has image in $A$ and $H_1$ has image $\{x_0\}$, and $H_0$ takes the boundary to $\{ x_0 \}$, we see that $[H] \in \pi_n(X,A,x_0)$, and moreover $\partial([H]) \simeq f$. Therefore, $[f] \in \im(\partial)$, and $\im(\partial) = \ker(i_*)$. 
\end{proof}
\begin{Def}
	A pair $(X,A)$ with basepoint $x_0$ is said to be $n$-connected if $\pi_i(X,A) = 0$ for all $i \le n$. 
\end{Def}
\begin{Lem}
	A pair $(X,A)$ is $n$-connected if and only if $\pi_i(A) \xr{i_*} \pi_i(X)$ is an isomorphism for $i < n$ and a surjection for $i = n$. 
\end{Lem}
\begin{proof}
	Use the long exact sequence in homotopy. 
\end{proof}
\begin{exercise}{}{}
Let $X$ be a path-connected space, and $CX$ the cone on $X$. Show that 
	\[
\pi_n(CX,X,X_0) \cong \pi_{n-1}(X,x_0)
	\]
	for $n \ge 1$. 
\end{exercise}
\section{Cofibrations and the homotopy extension property}
\begin{Def}
Let $\cat C$ be a class of topological spaces. A map $i \colon A \to X$ has the homotopy extension property (HEP) if, for every $Y \in \cat C$, the following extension property has a solution\sidenote{Here $i_0(x) = (x,0)$.}
% https://q.uiver.app/?q=WzAsNSxbMCwwLCJBIl0sWzAsMSwiWCJdLFsxLDAsIkEgXFx0aW1lcyBJIl0sWzEsMSwiWCBcXHRpbWVzIEkiXSxbMiwyLCJYIl0sWzAsMSwiaSIsMl0sWzAsMiwiIiwwLHsic3R5bGUiOnsidGFpbCI6eyJuYW1lIjoiaG9vayIsInNpZGUiOiJ0b3AifX19XSxbMSwzLCJpXzAiLDIseyJzdHlsZSI6eyJ0YWlsIjp7Im5hbWUiOiJob29rIiwic2lkZSI6InRvcCJ9fX1dLFsyLDMsImYgXFx0aW1lcyBcXHRleHR7aWR9Il0sWzIsNCwiSCIsMCx7ImN1cnZlIjotM31dLFsxLDQsImYiLDIseyJjdXJ2ZSI6M31dLFszLDQsIlxcZXhpc3RzIFxcdGlsZGUgSCIsMCx7InN0eWxlIjp7ImJvZHkiOnsibmFtZSI6ImRhc2hlZCJ9fX1dXQ==
\[\begin{tikzcd}
	A & {A \times I} \\
	X & {X \times I} \\
	&& Y
	\arrow["i"', from=1-1, to=2-1]
	\arrow["i_0", hook, from=1-1, to=1-2]
	\arrow["{i_0}"', hook, from=2-1, to=2-2]
	\arrow["{i \times \text{id}}", from=1-2, to=2-2]
	\arrow["H", curve={height=-18pt}, from=1-2, to=3-3]
	\arrow["f"', curve={height=18pt}, from=2-1, to=3-3]
	\arrow["{\exists \tilde H}", dashed, from=2-2, to=3-3]
\end{tikzcd}\]
A map $f \colon A \to X$ is a cofibration if it has the HEP with respect to all spaces $Y$.\sidenote{We will see later that cofibrations are always inclusions, and, if $X$ is Hausdorff, are always closed maps.}
\end{Def}
\begin{Rem}
Note that we do not ask that $\tilde H$ is unique. 
\end{Rem}
\begin{Rem}\label{Rem:hep_adjoint}
If we are in a 'nice' category of topological spaces (see CREF), which we always assume, then we have an adjunction
\[
\Hom(X,\Hom(Y,Z)) \cong \Hom(X \otimes Y,Z)
\]
of topological spaces, where $\Hom(Y,Z)$ is given the compact open topology. Writing, $Z^Y \coloneqq \Hom(Y,Z)$, the homotopy extension property admits a reformulation in the following diagram
% https://q.uiver.app/?q=WzAsNCxbMCwwLCJBIl0sWzEsMCwiWV5JIl0sWzAsMSwiWCJdLFsxLDEsIlkiXSxbMCwxLCJoIl0sWzAsMiwiaSIsMl0sWzIsMywiZiIsMl0sWzEsMywicCJdLFsyLDEsIlxcdGlsZGUgaCIsMCx7InN0eWxlIjp7ImJvZHkiOnsibmFtZSI6ImRhc2hlZCJ9fX1dXQ==
\[\begin{tikzcd}
	A & {Y^I} \\
	X & Y
	\arrow["h", from=1-1, to=1-2]
	\arrow["i"', from=1-1, to=2-1]
	\arrow["f"', from=2-1, to=2-2]
	\arrow["p", from=1-2, to=2-2]
	\arrow["{\exists \tilde h}", dashed, from=2-1, to=1-2]
\end{tikzcd}\]
where $p \colon Y^I \to Y$ is the evaluation at 0 map. It is often easier to work with this equivalent diagram. 
\end{Rem}
\begin{exercise}{}{}
Let $(X,A)$ have the HEP, and assume moreover that $i \colon A \to X$ is a retract up to homotopy. Show that $A$ is a retract of $X$. 
\end{exercise}
\begin{Lem}
	Let $J = [0,1]$. 
	\begin{enumerate}[label=(\roman*)]
		\item The inclusion $i_0 \colon X \to X \times J$ has the homotopy extension property for all $Y$. 
		\item The inclusion $i_0 \colon X \to CX$ has the homotopy extension property for all $Y$.   
	\end{enumerate}
\end{Lem}
\begin{proof}
	The proof in both cases is very similar; we do the first case in some detail. We are claiming there exists a lift $\tilde H$ in the following diagram:
	\[\begin{tikzcd}
	X & {X \times I} \\
	X \times J & {X \times J \times I} \\
	&& Y
	\arrow["i"', from=1-1, to=2-1]
	\arrow["i_0", hook, from=1-1, to=1-2]
	\arrow["{i_0}"', hook, from=2-1, to=2-2]
	\arrow["{i \times \text{id}}", from=1-2, to=2-2]
	\arrow["H", curve={height=-18pt}, from=1-2, to=3-3]
	\arrow["f"', curve={height=18pt}, from=2-1, to=3-3]
	\arrow["{\exists \tilde H}", dashed, from=2-2, to=3-3]
\end{tikzcd}\]
Geometrically, we will do this in two parts: we will define a map that "stacks" the two intervals on top of each other, i.e., we construct a map $G \colon X \times J \times I \to X \times [0,2]$. We will then do $H$ on one part of the cylinder, and $f$ on the remaining part. 

For the first part, let $G \colon X \times J \times I \to X \times [0,2]$ be defined as\sidenote{To see what is going on it is worth testing some cases and drawing pictures. For example, when $t = 0$ we have $G(x,0,s) =  (x,0)$. When $t = 1$ we have $G(x,1,s) = (x,1+s)$. When $s = 0$ we have $G(x,t,0) = (x,t)$ and when $s = 1$ we have $G(x,t,1) = (x,2t)$.}
\[
G(x,t,s) = (x,t(1+s)). 
\]
We then define $F \colon X \times [0,2] \to Y$ by
\[
F(x,k) = \begin{cases}
	f(x,k) & 0 \le k \le 1 \\
	H(x,k/2) & 1 \le k \le 2. 
\end{cases}
\]
Putting these together and defining $\tilde H \coloneqq F \circ G$, we see that\sidenote{Again, it is worthwhile to consider some cases. For example, if $t = 0$, then $(1-t)(1+s) = (1+s) \ge 1$ for all $s$, so $\tilde H((x,0),s) = H(x,s)$. At the other extreme, if $t =1$, then $(1-t)(1+s) = 0 \le 1$ for all $s$, so $\tilde H((x,1),s) = f(x,1)$. }
\[
\tilde H((x,t),s) = \begin{cases}
f(x,1-(1-t)(1+s)), & (1-t)(1+s) \le 1 \\
H(x,(1-t)(1+s)-1), & (1-t)(1+s) \ge 1. 
\end{cases}
\]
One verifies directly that this gives the required extension. 
\end{proof}
\begin{Rem}
We recall that given a map $f \colon X \to Y$, the mapping cylinder (see \Cref{fig_mapping_cylinder}) is the pushout\begin{marginfigure}[5\baselineskip] \centering\includegraphics[scale = 0.2]{map_cylinder.png}\caption{The mapping cylinder.}\label{fig_mapping_cylinder}\end{marginfigure}
% https://q.uiver.app/?q=WzAsNCxbMCwwLCJYIl0sWzAsMSwiWSJdLFsxLDEsIk1fZiJdLFsxLDAsIlggXFx0aW1lcyBJIl0sWzAsMSwiZiIsMl0sWzEsMl0sWzAsMywiaV8xIl0sWzMsMl0sWzIsMCwiIiwxLHsic3R5bGUiOnsibmFtZSI6ImNvcm5lci1pbnZlcnNlIn19XV0=
\[\begin{tikzcd}
	X & {X \times I} \\
	Y & {M_f}
	\arrow["f"', from=1-1, to=2-1]
	\arrow[from=2-1, to=2-2]
	\arrow["{i_0}", from=1-1, to=1-2]
	\arrow[from=1-2, to=2-2]
	\arrow["\ulcorner"{anchor=center, pos=0.125, rotate=180}, draw=none, from=2-2, to=1-1]
\end{tikzcd}\]
In formulas, 
\[
M_f = ((X \times I) \coprod Y)/((0,x) \sim f(x), \hspace{1mm} \forall x \in X)
\]

Note that $M_f$ deformation retracts on $Y$ by sliding each point $(x,t) \in M_f$ to the end-point. Note that we have a natural map $j \colon X \to M_f$ sending $x$ to $(x,1)$.
\end{Rem}
\begin{Lem}
The map $j \colon X \to M_f$ has the HEP for all spaces $Y$.
\end{Lem}
\begin{proof}
The proof is similar to the previous lemma; one just has to modify the end point by defining
\[
\tilde H|_{Y \times I}(y,s) = f(y,0). 
\]
\end{proof}
\begin{Cor}\label{cor:spheres_are_cofibrations}
The inclusion $S^{n-1} \to D^n$ is a cofibration. 
\end{Cor}
\begin{proof}
	Simply note that $D^n \simeq CS^{n-1}$. 
\end{proof}
There is a universal test space for cofibrations. 
\begin{Prop}\label{prop:universal_cofibration}
	Let $i \colon A \to X$, and let $M_i$ be the mapping cylinder. Then $i \colon A \to X$ is a cofibration if and only if there exists a map $r \colon X \times I \to M_i$ making the diagram
% https://q.uiver.app/?q=WzAsNSxbMCwwLCJBIl0sWzAsMiwiWCJdLFsyLDIsIk1faSJdLFsyLDAsIkEgXFx0aW1lcyBJIl0sWzEsMSwiWCBcXHRpbWVzIEkiXSxbMCwxLCJpIiwyXSxbMSwyXSxbMCwzLCJpXzAiXSxbMywyXSxbMiwwLCIiLDEseyJzdHlsZSI6eyJuYW1lIjoiY29ybmVyLWludmVyc2UifX1dLFsxLDQsImlfMCIsMl0sWzMsNCwiaSBcXHRpbWVzIFxcdGV4dHtpZH0iLDJdLFs0LDIsIlxcZXhpc3RzIHIiLDAseyJzdHlsZSI6eyJib2R5Ijp7Im5hbWUiOiJkYXNoZWQifX19XV0=
\[\begin{tikzcd}
	A && {A \times I} \\
	& {X \times I} \\
	X && {M_i}
	\arrow["i"', from=1-1, to=3-1]
	\arrow[from=3-1, to=3-3]
	\arrow["{i_0}", from=1-1, to=1-3]
	\arrow[from=1-3, to=3-3]
	\arrow["\ulcorner"{anchor=center, pos=0.125, rotate=180}, draw=none, from=3-3, to=1-1]
	\arrow["{i_0}"', from=3-1, to=2-2]
	\arrow["{i \times \text{id}}"', from=1-3, to=2-2]
	\arrow["{\exists r}", dashed, from=2-2, to=3-3]
\end{tikzcd}\]
commute. 
\end{Prop}
\begin{proof}
If $i$ is a cofibration, then the map $r$ exists as a consequence of the HEP. 

For the other direction, if $r$ exists, then for any maps $f \colon X \to Y$ and $H \colon A \times I \to Y$ making the obvious diagram commute, the universal property of the pushout gives us a map $H' \colon M_i \to Y$. Then let $\tilde H = H' \circ r$, and we are done. 
\end{proof}
\begin{Cor}\label{cor:cofibration_retract}
If $A \subseteq X$, then $I \colon A \to X$ is a cofibration if and only if $X \times I$ is a retract of $M_i = X \times \{ 0 \} \cup A \times I$. 
\end{Cor}
\begin{Cor}
A cofibration $i \colon A \to X$ is an injection. If $X$ is Hausdorff, then $i(A)$ is closed in $X$. 
\end{Cor}
\begin{proof}
Let $J \colon A \times I \to M_i$ be the canonical map (arising from the definition of $M_i$ as a pushout). Then, $J(a,1) = r(i(a),1)$, and observe that $J\mid_{A \times \{ 1 \}}$ is the identity, as it is the top of the mapping cylinder. So, $i(a) \ne i(a')$ if $a \ne a'$, i.e., $i$ is injective. 

Because $i \colon A \to X$ is a cofibration, so is $i(A) \to X$. Hence $X \times I$ retracts onto $X \times \{ 0 \} \cup i(A) \times I$ (\Cref{cor:cofibration_retract}). For a Hausdorff space, the image of a retract is closed, and so $X \times \{ 0 \} \cup i(A) \times I$ is a closed subspace of $X \times I$. Intersecting with $X \times \{ 1 \}$, we see that $i(A) \times \{ 1 \}$ is closed in $X \times \{ 1 \}$, i.e, $i(A)$ is closed in $X$.  
\end{proof}
The following (rather pathological) example shows that $i$ is not always a closed map if $X$ is not Hausdorff. 
\begin{exercise}{}{}
Let $A = \{ a \}$ and $X = \{ a , b\}$ with the trivial topology. Show that the inclusion $A \to X$ is a cofibration whose image is not closed. 
\end{exercise}
\begin{Rem}
The next goal is to show that CW-complexes $(X,A)$ are always cofibrations. The key is the following exercise. 
\end{Rem}
\begin{exercise}{}{}
\begin{enumerate}[label=(\alph*)]
  	\item Suppose $\{ (X_i,A_i)\}$ are a collection of spaces satisfying the HEP, then so does $\{ (\coprod X_i, \coprod A_i) \}$.
  	\item Suppose $(X,A)$ satisfies the HEP, and $f \colon A \to B$ is a continuous map. Let $Y = X \cup_f B$ be the pushout, then $(Y,B)$ satisfies the HEP. 
  	\item Suppose $A = X_0 \subseteq X_1 \subseteq \cdots \subseteq X_n \subseteq X_{n+1} \subseteq \cdots$.

  	Let $X = \colim X_i$. If each $(X_i,X_{i-1})$ satisfies the HEP, then so does $(X,A)$. 
  \end{enumerate}  
\end{exercise}

\begin{Thm}
A relative $CW$-complex $(X,A)$ satisfies the HEP. 
\end{Thm}
\begin{proof}
Using \Cref{cor:spheres_are_cofibrations} and the previous exercise we see that $(S^{n-1},D^n)$ satisfies the HEP $\implies (\coprod S^{n-1},\coprod D^n)$ satisfies the HEP. Inductively, $(X_{n-1},A)$ satisfies the HEP and by the exercise $(X,A)$ satisfies the HEP. 
\end{proof}
\begin{Rem}
One can also prove this directly by constructing a deformation retract $r \colon X \times I \to X \times \{ 0 \} \cup A \times I$. 
\end{Rem}
\begin{Rem}
One can consider the following question: Suppose that $A \subset X$ with $A$ contractible, then is $X \simeq X/A$? Surprisingly, this is not true in general. Indeed, let $A = S^1 \setminus \{ (1,0) \}$ and consider $A \to S^1$. Then $S^1/A \cong T$, the $T = \{ a ,b \}$ the two point space with open sets $\emptyset, \{ a \}, \{ a, b\}$ (this is the Sierpi\'nski space). One can check that this space is contractible.\sidenote{See \url{https://math.stackexchange.com/a/264789/64273}.} The exact condition we need is that $A \to X$ is a cofibration. 
\end{Rem}
\begin{Def}
A contracting homotopy is a map $H \colon X \times I \to X$ such that $H(x,0) = \text{id}_X$ and $H(x,1) = c_{x_0}$, the constant map at $x_0$. 
\end{Def}
\begin{Prop}\label{prop:quotient_contractible}
Suppose $A \subseteq X$ and $x_0 \in A$. Suppose there exists a map $H \colon X \times I \to X$ such that $H \mid_{X \times \{ 0 \}} = \text{id}_X$ and $H \mid_{A \times I}$ has image in $A$ and is a contacting homotopy for $A$. Then $q \colon X \to X/A$ is a homotopy equivalence. 
\end{Prop}
\begin{proof}
We need to find $p \colon X/A \to X$ such that $q \circ p \simeq \text{id}_{X/A}$ and $p \circ q \simeq \text{id}_X$. The quotient map has a set-theoretic section given by 
\[
s(\overline x) = \begin{cases}
	x & x \not\in A\\
	x_0 & x \in A
\end{cases}
\]
Define $p \colon X/A \to X$ by the following diagram
% https://q.uiver.app/?q=WzAsNCxbMCwwLCJYIl0sWzEsMCwiWC9BIl0sWzIsMCwiWCJdLFsyLDEsIlgiXSxbMCwxLCJxIl0sWzEsMiwicyJdLFsyLDMsIkggXFxtaWRfe1ggXFx0aW1lcyBcXHsgMSBcXH19Il0sWzEsMywicCIsMl1d
\[\begin{tikzcd}
	X & {X/A} & X \\
	&& X
	\arrow["q", from=1-1, to=1-2]
	\arrow["s", from=1-2, to=1-3]
	\arrow["{H \mid_{X \times \{ 1 \}}}", from=1-3, to=2-3]
	\arrow["p"', from=1-2, to=2-3]
\end{tikzcd}\]
Assume for a moment that $p$ is continuous. Then $p \circ q = H\mid_{X \times \{ 1 \}}$, and so $H$ gives a homotopy between $\text{id}_X$ and $p \circ q = H \mid_{X \times \{ 1 \}}$. Likewise, if we define $G$ by 
% https://q.uiver.app/?q=WzAsNCxbMCwwLCJYL0EgXFx0aW1lcyBJIl0sWzEsMCwiWCBcXHRpbWVzIEkiXSxbMiwwLCJYIl0sWzIsMSwiWC9BIl0sWzAsMSwicyBcXHRpbWVzIFxcdGV4dHtpZH0iXSxbMSwyLCJIIl0sWzIsMywicSJdLFswLDMsIkciLDJdXQ==
\[\begin{tikzcd}
	{X/A \times I} & {X \times I} & X \\
	&& {X/A}
	\arrow["{s \times \text{id}}", from=1-1, to=1-2]
	\arrow["H", from=1-2, to=1-3]
	\arrow["q", from=1-3, to=2-3]
	\arrow["G"', from=1-1, to=2-3]
\end{tikzcd}\]
and assume that $G$ is continuous, then 
\[
G(\overline x,1) = q \circ (H\mid_{X \times \{ 1 \}} \circ s) = q \circ p,
\]
so that $G$ is a homotopy between $\text{id}_{X/A}$ and $q \circ p$. To see that $p$ is continuous, let $U \subset X$ be open, then 
\[
q^{-1}p^{-1}(U) = (p \circ q)^{-1}(U) = (H \mid_{X \times \{ 1\}})^{-1}(U)
\]
is open in $X$ by the continuity of $H \mid_{X \times \{ 1\}}$, hence $p^{-1}(U)$ is open in $X/A$ by the definition of the quotient topology, and so $p$ is continuous. We leave the proof of continuity of $G$ to the reader. 
\end{proof}
\begin{Thm}
Let $A \subseteq X$ be a subspace with $A$ contractible. Suppose that the inclusion $i \colon A \to X$ is a cofibration, then $X \to X/A$ is a homotopy equivalence. 
\end{Thm}
\begin{proof}
	Let $h \colon A \to I \to A$ be a contracting homotopy. Let $H \colon A \times I \to X$ be the composition of $h$ with the inclusion map of $A$ into $X$, i.e., the following diagram commutes:
	\[\begin{tikzcd}
	A & {A \times I} \\
	X & {X \times I} \\
	&& X
	\arrow["i"', from=1-1, to=2-1]
	\arrow["i_0", hook, from=1-1, to=1-2]
	\arrow["{i_0}"', hook, from=2-1, to=2-2]
	\arrow["{i \times \text{id}}", from=1-2, to=2-2]
	\arrow["H", curve={height=-18pt}, from=1-2, to=3-3]
	\arrow["\text{id}_X"', curve={height=18pt}, from=2-1, to=3-3]
	\arrow["{\exists \tilde H}", dashed, from=2-2, to=3-3]
\end{tikzcd}\]
By the HEP, the dotted map $\tilde H$ exists as in the diagram. This map satisfies the conditions of \Cref{prop:quotient_contractible}:
\begin{enumerate}[label=(\roman*)]
	\item $\tilde H \colon X \times \{ 0 \} \to X$ is the identity. 
	\item $\tilde H(A \times I) = H(A \times I) = h(A \times I) \subseteq A$.
	\item $\tilde H(A \times \{ 1 \}) = x_0$. 
\end{enumerate}
Therefore, $q \colon X \to X/A$ is a homotopy equivalence, as claimed. 
\end{proof}
\begin{exercise}[label=ex:cofibratonpushout]{Cofibrations are pushout closed.}{}
Let $i \colon A \to X$ be a cofibration, and $g \colon A \to B$ any map, then the induced map $B \to B \cup_g X$ is a fibration. 
\end{exercise}
\section{Fibrations and the homotopy lifting property}
The dual notion of a cofibration is a fibration, where the homotopy extension property is replaced by the homotopy lifting property. 
\begin{Def}
	Let $\mathcal{E}$ be a class of topological spaces. Assume that $p \colon E \to B$ is a continuous map, then we say that $p$ has the homotopy lifting property (with respect to $\cal E$) if for every $X \in \cal E$, and map $f \colon X \to E$ and every homotopy $H \colon X \times I \to B$ that begins with $p \circ f$, we can lift it to a homotopy $\tilde H \colon X \times I \to E$ that begins with $f$, i.e., $p \circ \tilde H = H$ and $\tilde H(x,0) = f(x)$. In a diagram, we require the lift $\tilde H$ in the following:
% https://q.uiver.app/?q=WzAsNCxbMSwwLCJFIl0sWzEsMSwiQiJdLFswLDAsIlgiXSxbMCwxLCJYIFxcdGltZXMgSSJdLFswLDEsInAiXSxbMiwwLCJmIl0sWzMsMSwiSCIsMl0sWzIsMywiaV8wIiwyLHsic3R5bGUiOnsidGFpbCI6eyJuYW1lIjoiaG9vayIsInNpZGUiOiJib3R0b20ifX19XSxbMywwLCJcXGV4aXN0cyBcXHRpbGRlIEgiLDAseyJzdHlsZSI6eyJib2R5Ijp7Im5hbWUiOiJkYXNoZWQifX19XV0=
\[\begin{tikzcd}
	X & E \\
	{X \times I} & B
	\arrow["p", from=1-2, to=2-2]
	\arrow["f", from=1-1, to=1-2]
	\arrow["H"', from=2-1, to=2-2]
	\arrow["{i_0}"', hook', from=1-1, to=2-1]
	\arrow["{\exists \tilde H}", dashed, from=2-1, to=1-2]
\end{tikzcd}\]
If $\cal E$ is the class of all topological spaces, then $p$ is called a (Hurewicz) fibration, while if $\mathcal{E} = \{ I^{n} \}$ (or equivalently, the class of CW-complexes), then $p$ is called a Serre fibration.  
\end{Def}
\begin{Rem}
As in \Cref{Rem:hep_adjoint}, there is an equivalent way to present the homotopy lifting property: we ask for the lift $\tilde h$ as shown in the following
% https://q.uiver.app/?q=WzAsNSxbMiwxLCJFIl0sWzIsMiwiQiJdLFsxLDEsIkVeSSJdLFsxLDIsIkJeSSJdLFswLDAsIlgiXSxbMCwxLCJwIl0sWzIsMCwiZXZfMCJdLFszLDEsImV2XzAiLDJdLFsyLDMsInBfKiIsMl0sWzQsMiwiXFxleGlzdHMgXFx0aWxkZSBoIiwwLHsic3R5bGUiOnsiYm9keSI6eyJuYW1lIjoiZGFzaGVkIn19fV0sWzQsMywiaCIsMix7ImN1cnZlIjoyfV0sWzQsMCwiZiIsMCx7ImN1cnZlIjotMn1dXQ==
\[\begin{tikzcd}
	X \\
	& {E^I} & E \\
	& {B^I} & B
	\arrow["p", from=2-3, to=3-3]
	\arrow["{ev_0}", from=2-2, to=2-3]
	\arrow["{ev_0}"', from=3-2, to=3-3]
	\arrow["{p_*}"', from=2-2, to=3-2]
	\arrow["{\exists \tilde h}", dashed, from=1-1, to=2-2]
	\arrow["h"', curve={height=12pt}, from=1-1, to=3-2]
	\arrow["f", curve={height=-12pt}, from=1-1, to=2-3]
\end{tikzcd}\]
This makes it clear how the homotopy lifting property is dual to the homotopy extension property. 
\end{Rem}
\begin{Rem}
	We can also talk about the homotopy lifting property with respect to a pair $(X,A)$: namely, a map $p \colon E \to B$ has the homotopy lifting property with respect to a pair $(X,A)$ if each homotopy $H \colon X \times I \to B$ lifts to a homotopy $\tilde H \colon X \times I \to E$ which agrees with a given homotopy $H_A$ on $A \times I$. In a diagram, we ask for the lift $\tilde H$ in the following:
% https://q.uiver.app/?q=WzAsNCxbMSwwLCJFIl0sWzEsMSwiQiJdLFswLDAsIlggXFxjdXAgKEEgXFx0aW1lcyBJKSJdLFswLDEsIlggXFx0aW1lcyBJIl0sWzAsMSwicCJdLFsyLDAsImYgXFxjdXAgSF9BIl0sWzMsMSwiSCIsMl0sWzIsMywiaV8wIiwyLHsic3R5bGUiOnsidGFpbCI6eyJuYW1lIjoiaG9vayIsInNpZGUiOiJib3R0b20ifX19XSxbMywwLCJcXGV4aXN0cyBcXHRpbGRlIEgiLDAseyJzdHlsZSI6eyJib2R5Ijp7Im5hbWUiOiJkYXNoZWQifX19XV0=
\[\begin{tikzcd}
	{X \cup (A \times I)} & E \\
	{X \times I} & B
	\arrow["p", from=1-2, to=2-2]
	\arrow["{f \cup H_A}", from=1-1, to=1-2]
	\arrow["H"', from=2-1, to=2-2]
	\arrow["{i_0}"', hook', from=1-1, to=2-1]
	\arrow["{\exists \tilde H}", dashed, from=2-1, to=1-2]
\end{tikzcd}\]
\end{Rem}
\begin{Thm}
	The following are equivalent:
	\begin{enumerate}[label=(\roman*)]
		\item $p$ is a Serre fibration. 
		\item $p$ has the homotopy lifting property with respect to all $n$-discs $D^n$. 
		\item $p$ has relative homotopy property with respect to all pairs $(D^n,S^{m-1})$
		\item $p$ has the relative homotopy property with respect to all CW-pairs $(X,A)$. 
	\end{enumerate}
\end{Thm}
\begin{proof}[Proof sketch]
	$(i) \implies (ii)$ is immediate from the definitions. 

	$(ii) \implies (iii)$ follows because the pairs $(D^n \times I,D^n \times \{ 0 \})$ and $(D^n \times I,D^n \times \{ 0 \} \cup S^{n-1} \times I)$ are homeomorphic. 

	$(iii) \implies (iv)$ by induction over the skeleton of $X$; one reduces to the case (iii). 

	$(iv) \implies (i)$ by taking $A = \emptyset$. \qedhere



\end{proof}
\begin{exercise}{}{}
Show that the composition of fibrations is a fibration. 
\end{exercise}

\begin{Def}
We recall the construction of pullbacks in topological spaces: given maps $p \colon E \to B$ and $f \colon B' \to B$, we let 
\[
E' = \{(b',e) \in B' \times E \mid p(e) = f(b') \}.
\]
This comes with natural projection maps $f' \colon E' \to E$ and $p' \colon E' \to B'$. Then $E'$ is the pull-back in topological spaces, and so we often also denote it by $f^*E$. 
\end{Def}
The following is dual to \Cref{ex:cofibratonpushout}.
\begin{Lem}
If $p \colon E \to B$ satisfies the HLP with respect to the class $\cal E$, then so does $p' \colon E' \to B'$. 
\end{Lem}
\begin{proof}
Consider the following diagram:
% https://q.uiver.app/?q=WzAsNixbMiwwLCJFIl0sWzIsMSwiQiJdLFswLDAsIlgiXSxbMCwxLCJYIFxcdGltZXMgSSJdLFsxLDAsIkUnIl0sWzEsMSwiQiciXSxbMCwxLCJwIl0sWzIsMywiaV8wIiwyLHsic3R5bGUiOnsidGFpbCI6eyJuYW1lIjoiaG9vayIsInNpZGUiOiJib3R0b20ifX19XSxbNSwxLCJmIiwyXSxbNCwwLCJmJyJdLFs0LDUsInAnIiwyXSxbNCwxLCIiLDEseyJzdHlsZSI6eyJuYW1lIjoiY29ybmVyIn19XSxbMiw0XSxbMyw1XV0=
\[\begin{tikzcd}
	X & {E'} & E \\
	{X \times I} & {B'} & B
	\arrow["p", from=1-3, to=2-3]
	\arrow["{i_0}"', hook', from=1-1, to=2-1]
	\arrow["f"', from=2-2, to=2-3]
	\arrow["{f'}", from=1-2, to=1-3]
	\arrow["{p'}"', from=1-2, to=2-2]
	\arrow["\lrcorner"{anchor=center, pos=0.125}, draw=none, from=1-2, to=2-3]
	\arrow[from=1-1, to=1-2]
	\arrow[from=2-1, to=2-2]
\end{tikzcd}\]
Because $p \colon E \to B$ satisfies the HLP, there is a lift $\tilde H' \colon X \times I \to E$ of $X \times I \to B$. Then, by the universal property of the pullback, we get a map $\tilde H \colon X \times I \to E'$ satisfying the desired properties. 
\end{proof}
\begin{Def}
	If $p \colon E \to B$ is a fibration, then $F \coloneqq p^{-1}(\ast)$ is called the fiber, $E$ is called the total space, and $B$ is the base space. We write this as 
	\[
F \to E \to B.
	\]
\end{Def}
\begin{Exa}\label{ex:loops_fibration}
	Given a based space $X$, let 
	\[
PX = \Hom_*(I,X) = \{ f \colon I \to X \mid f(0) = \ast \}
	\]
	be the space of paths starting at the base-point. Then $PX \xrightarrow{p_1} X$ is a fibration with fiber $\Omega X$, the loop space in $X$ (i.e., $f(0) = f(1) = \ast$). To see this, consider our test diagram, where we must show that $\tilde H$ exists: 
	\[\begin{tikzcd}
	A & PX \\
	{A \times I} & X
	\arrow["p_1", from=1-2, to=2-2]
	\arrow["g", from=1-1, to=1-2]
	\arrow["H"', from=2-1, to=2-2]
	\arrow["{i_0}"', hook', from=1-1, to=2-1]
	\arrow["{\exists \tilde H}", dashed, from=2-1, to=1-2]
\end{tikzcd}\]
Note that for each $a \in A$, $g(a)$ is a path in $X$ which ends at $p_1g(a) = H(a,0)$. This point is the start of the path $H(a,-)$. % https://q.uiver.app/?q=WzAsMyxbMCwxLCJcXGJ1bGxldCJdLFsxLDAsIlxcc3RhY2tyZWx7SChhLDApfXtcXGJ1bGxldH0iXSxbMiwxLCJcXGJ1bGxldCJdLFswLDEsImcoYSkiLDAseyJjdXJ2ZSI6MSwic3R5bGUiOnsidGFpbCI6eyJuYW1lIjoibW9ubyJ9fX1dLFsxLDIsIkgoYSwtKSIsMCx7ImN1cnZlIjoxLCJzdHlsZSI6eyJ0YWlsIjp7Im5hbWUiOiJtb25vIn19fV1d
\[\begin{tikzcd}
	& {\stackrel{H(a,0)}{\bullet}} \\
	\bullet && \bullet
	\arrow["{g(a)}", curve={height=6pt}, start anchor={[xshift=-1ex]}, end anchor={[xshift=1ex]},from=2-1, to=1-2]
	\arrow["{H(a,-)}", curve={height=6pt}, end anchor={[xshift=1ex]}, from=1-2, to=2-3]
\end{tikzcd}\]
We will define $\tilde H(a,s)(t)$ to be a path running along $g(a)$ and then part-way along $H(a,-)$ ending at $H(a,s)$. In symbols, 
\[
\tilde H(a,s)(t) = \begin{cases}
	g(a)((1+s)t) & 0 \le t \le 1/(1+s) \\
	H(a,(1+s)t-1) & 1/(1+s) \le t \le 1. 
\end{cases}
\]
Then $\tilde H(a,0) = g(a)$ and $p_1\tilde H(a,s) = \tilde H(a,s)(1) = H(a,s)$, as required. 

The same argument shows that there is a fibration 
\[
p_*Y \to Y^I \xrightarrow{p_1} Y
\]
where $p_*Y$ is the space of paths with end-point $\ast$. 
\end{Exa}
\begin{Def}
	Given $f \colon X \to Y$ the mapping path space $P_f$ (or mapping cocylinder), is the pullback of $f$ along $Y^I \xrightarrow{p_1}Y$, i.e., 
% https://q.uiver.app/?q=WzAsNCxbMCwwLCJQX2YiXSxbMCwxLCJYIl0sWzEsMCwiWV5JIl0sWzEsMSwiWSJdLFsyLDMsInBfMSJdLFsxLDMsImYiLDJdLFswLDEsInAnIiwyXSxbMCwyXSxbMCwzLCIiLDEseyJzdHlsZSI6eyJuYW1lIjoiY29ybmVyIn19XV0=
\[\begin{tikzcd}
	{P_f} & {Y^I} \\
	X & Y
	\arrow["{p_1}", from=1-2, to=2-2]
	\arrow["f"', from=2-1, to=2-2]
	\arrow["{p'}"', from=1-1, to=2-1]
	\arrow[from=1-1, to=1-2]
	\arrow["\lrcorner"{anchor=center, pos=0.125}, draw=none, from=1-1, to=2-2]
\end{tikzcd}\]
Note that $P_f \simeq X$. 
\end{Def}
\begin{Prop}
The map $p \colon P_f \to Y$ given by $p(x,\alpha) = \alpha(1)$ is a fibration. 
\end{Prop}
\begin{proof}
	This is very similar to \Cref{ex:loops_fibration}. Our test diagram is the following: 
		\[\begin{tikzcd}
	A & P_f \\
	{A \times I} & Y
	\arrow["p", from=1-2, to=2-2]
	\arrow["g", from=1-1, to=1-2]
	\arrow["H"', from=2-1, to=2-2]
	\arrow["{i_0}"', hook', from=1-1, to=2-1]
	\arrow["{\exists \tilde H}", dashed, from=2-1, to=1-2]
\end{tikzcd}\]
Note that $g(a) \in P_f \subset X \times Y^I$, so we can write $g(a) = (g_1(a),g_2(a))$. Here $g_1(a)$ maps via $f$ to the starting point of the path $g_2(a)$ and the commutativity of the  diagram implies that the endpoint of the path $g_2(a)$ is the starting point of $H(a,-)$. The lift $\tilde H$ will have two components. The $x$ component will be constant in $s$, i.e., $\tilde H_1(a,s) = g_1(a)$. Overall, we define
\[
\tilde H(a,s) = (g_1(a),\tilde H_2(a,s)(-)) \in P_f
\]
where\sidenote{Compare this to the formula in \Cref{ex:loops_fibration}.}
\[
\tilde H_2(a,s)(t) = \begin{cases}
	g_2(a)((1+s)t) & 0 \le t \le 1/(1+s) \\
	H(a,(1+s)t-1) & 1/(1+s) \le t \le 1. 
\end{cases}
\]
One check directly that $\tilde H(a,s)$ has the required properties. 
\end{proof}
As with the homotopy extension property, we have a universal test space. The details (which are dual to \Cref{prop:universal_cofibration}) are left to the reader. 
\begin{Prop}
	Let $f \colon E \to B$ be a continuous map, then $f$ is a fibration if and only if there exists $s \colon P_f \to E^I$ making the following diagram commute:
	\[\begin{tikzcd}
	P_f \\
	& {E^I} & E \\
	& {B^I} & B
	\arrow["f", from=2-3, to=3-3]
	\arrow["{ev_0}", from=2-2, to=2-3]
	\arrow["{ev_0}"', from=3-2, to=3-3]
	\arrow["{f_*}"', from=2-2, to=3-2]
	\arrow["{\exists s}", dashed, from=1-1, to=2-2]
	\arrow["\pi_{B_I}"', curve={height=12pt}, from=1-1, to=3-2]
	\arrow["\pi_E", curve={height=-12pt}, from=1-1, to=2-3]
\end{tikzcd}\]
where $\pi_{B_I}$ and $\pi_E$ are the projection maps coming from the construction of $P_f$ as a pullback. 
\end{Prop}
\begin{Rem}
One property of cofibrations that does not dualize to fibrations is that cofibrations are inclusions, but fibrations need not be surjective. Indeed, given $p \colon E \to B$ a fibration, then the composite
\[
E \xrightarrow{p} B \hookrightarrow B \coprod \ast
\]
is also a fibration, but is not surjective. 
\end{Rem}
\begin{Rem}
	We will want to talk about exact sequences where the terms appearing may not have a group structure, but are rather only sets with base-points. Therefore, given a sequence of functions
	\[
A \xrightarrow{f} B \xrightarrow{g} C
	\]
	of sets with base-points, we say that this is exact at $B$ if $f(A) = g^{-1}(c_0)$ where $c_0$ is the base-point of $C$. Note that if $A,B,C$ are groups with base-points the identity elements of the group, then exactness of sets corresponds to exactness of groups. 
\end{Rem}
\begin{Thm}\label{thm:fibration_ses}
	Let $p \colon E \to B$ be a fibration with fiber $F$ and $B$ path-connected. Let $Y$ be any space, then
	\[
[Y,F] \xrightarrow{i_*} [Y,E] \xrightarrow{p_*} [Y,B]
	\]
	is exact. 
\end{Thm}
\begin{proof}
	For one direction, it is clear that $p_*(i_*[g])) = 0$.

	 Suppose $f \in [Y,E]$ is such that $p_*[f] = [\text{const}]$, i.e., $p \circ f$ is null-homotopic. Let $G \colon Y \times I \to B$ be a null-homotopy, and let $H \colon Y \times I \to E$ be a solution to the lifting problem indicated in the following diagram, using that $p$ is a fibration:
	% https://q.uiver.app/?q=WzAsNCxbMCwwLCJZIFxcdGltZXMgXFx7MFxcfSJdLFswLDEsIlkgXFx0aW1lcyBJIl0sWzEsMSwiQiJdLFsxLDAsIkUiXSxbMSwyLCJHIiwyXSxbMywyLCJwIl0sWzAsMSwiaV8wIiwyLHsic3R5bGUiOnsidGFpbCI6eyJuYW1lIjoiaG9vayIsInNpZGUiOiJib3R0b20ifX19XSxbMCwzXSxbMSwzLCJIIiwwLHsic3R5bGUiOnsiYm9keSI6eyJuYW1lIjoiZGFzaGVkIn19fV1d
\[\begin{tikzcd}
	{Y \times \{0\}} & E \\
	{Y \times I} & B
	\arrow["G"', from=2-1, to=2-2]
	\arrow["p", from=1-2, to=2-2]
	\arrow["{i_0}"', hook', from=1-1, to=2-1]
	\arrow["f",from=1-1, to=1-2]
	\arrow["H", dashed, from=2-1, to=1-2]
\end{tikzcd}\]
Note now that $p \circ H(y,1) = G(y,1) = b_0$, so that $H(y,1) \in F \coloneqq p^{-1}(b_0)$. It follows that $[f] = i_*[H(-,1)]$. 
\end{proof}
We have an analogous result for cofibration. 
\begin{Thm}
	Let $i \colon A \to X$ be a cofibration, and $q \colon X \to X/A$ the quotient map. Let $Y$ be any path-connected space, then the sequence of pointed sets
	\[
[X/A,Y] \xrightarrow{q^*} [X,Y] \xrightarrow{i^*} [A,Y]
	\]
	is exact. 
\end{Thm}
\begin{proof}
	Again, one inclusion is clear: we have $i^*(g^*([g])) = [g \circ q \circ i] = [\text{const}]$. 

	Now suppose that $f \colon X \to Y$ is a map with $f\mid_{A} \colon A \to Y$ null-homotopic. Let $h \colon A \times I \to Y$ be a hull-homotopy, and let $F \colon X \times I \to Y$ be the extension as shown in the following diagram:
	\[\begin{tikzcd}
	A & {A \times I} \\
	X & {X \times I} \\
	&& Y
	\arrow["i"', from=1-1, to=2-1]
	\arrow["i_0", hook, from=1-1, to=1-2]
	\arrow["{i_0}"', hook, from=2-1, to=2-2]
	\arrow["{i \times \text{id}}", from=1-2, to=2-2]
	\arrow["H", curve={height=-18pt}, from=1-2, to=3-3]
	\arrow["f"', curve={height=18pt}, from=2-1, to=3-3]
	\arrow["{F}", dashed, from=2-2, to=3-3]
\end{tikzcd}\]
Let $f' \coloneqq F(-,1)$. Then, $f \sim f'$ and $f'(A) = F(A,1) = y_0$. By the universal property of the quotient, we can find $g \colon X/A \to Y$ making the following diagram commute:
% https://q.uiver.app/?q=WzAsMyxbMCwwLCJYIl0sWzEsMCwiWSciXSxbMCwxLCJYL0EiXSxbMCwyLCJxIiwyXSxbMCwxLCJmJyJdLFsyLDEsImcnIiwyLHsic3R5bGUiOnsiYm9keSI6eyJuYW1lIjoiZGFzaGVkIn19fV1d
\[\begin{tikzcd}
	X & {Y} \\
	{X/A}
	\arrow["q"', from=1-1, to=2-1]
	\arrow["{f'}", from=1-1, to=1-2]
	\arrow["{g'}"', dashed, from=2-1, to=1-2]
\end{tikzcd}\]
Therefore $[f] = [f'] = q^*[g']$.
\end{proof}
As an extension of \Cref{thm:fibration_ses} we have the following. 
\begin{Thm}\label{thm:les_fibration}
	Given a (Serre) fibration $p \colon E \to B$, and base points $b \in B$ and $e \in F \coloneqq f^{-1}(b)$, then there is an isomorphism $p_* \colon \pi_n(E,F,e) \xrightarrow{\simeq} \pi_n(B,b)$ for all $n \ge 1$. Hence, if $B$ is path-connected, there is a long exact sequence of homotopy groups
	\[
	\begin{split}
\cdots \pi_n(F,e) \to \pi_n(E,e) \xrightarrow{p_*} \pi_n(B,b) \to \pi_{n-1}(F,e) \to \cdots \\ \cdots \to \pi_0(E,e) \to 0. 
\end{split}
	\]
\end{Thm}
\begin{proof}
	We first show that $p_*$ is surjective. Let $[f] \in \pi_n(B,b)$, represented by a map $f \colon (I^n, \partial I^n) \to (B,b)$. Note that $I^{n-1} \times \{ 0 \} \subseteq \partial I^n$, so we can form the diagram
% https://q.uiver.app/?q=WzAsNCxbMCwxLCJJXm4iXSxbMSwxLCJCIl0sWzEsMCwiRSJdLFswLDAsIklee24tMX0gXFx0aW1lcyBcXHsgMCBcXH0iXSxbMCwxLCJmIiwyXSxbMiwxLCJwIl0sWzMsMF0sWzMsMiwiXFxhc3QiXSxbMCwyLCJmIiwwLHsic3R5bGUiOnsiYm9keSI6eyJuYW1lIjoiZGFzaGVkIn19fV1d
\[\begin{tikzcd}
	{I^{n-1} \times \{ 0 \}} & E \\
	{I^n} & B
	\arrow["f"', from=2-1, to=2-2]
	\arrow["p", from=1-2, to=2-2]
	\arrow[from=1-1, to=2-1]
	\arrow["\ast", from=1-1, to=1-2]
	\arrow["\tilde f", dashed, from=2-1, to=1-2]
\end{tikzcd}\]
where the lift $\tilde f$ exists because $p$ is a Serre fibration. Because $f(\partial I^n) = b$, we have $\tilde f(\partial I^n) \subseteq F$. So $\tilde f $ represents an element of $\pi_n(E,F,e)$ with $p_*([\tilde f]) = [p \circ \tilde f] =[f]$. 

To show injectivity, let $\tilde f_0,\tilde f_1 \colon (I^n,\partial I^n,J^{n-1}) \to (E,F,e)$ be such that $p_*(\tilde f_0) = p_*(\tilde f_1)$. Let $H \colon (I^n 
\times I,\partial I^n \times I) \to (B,b)$ be a homotopy from $p\tilde f_0$ to $p \tilde f_1$. We can find a lift in the following diagram:
% https://q.uiver.app/?q=WzAsNCxbMCwxLCJJXm4gXFx0aW1lcyBJIl0sWzEsMSwiQiJdLFsxLDAsIkUiXSxbMCwwLCJXIl0sWzAsMSwiSCIsMl0sWzIsMSwicCJdLFszLDBdLFszLDIsImYiXSxbMCwyLCJcXHRpbGRlIEgiLDAseyJzdHlsZSI6eyJib2R5Ijp7Im5hbWUiOiJkYXNoZWQifX19XV0=
\[\begin{tikzcd}
	W & E \\
	{I^n \times I} & B
	\arrow["H"', from=2-1, to=2-2]
	\arrow["p", from=1-2, to=2-2]
	\arrow[from=1-1, to=2-1]
	\arrow["f", from=1-1, to=1-2]
	\arrow["{\tilde H}", dashed, from=2-1, to=1-2]
\end{tikzcd}\]
where $W = I^n \times \{ 0 \} \cup I^n \times \{ 1 \} \cup \partial I^n \times I$, and $f$ is $\tilde f_0$ on $I^n \times \{ 0 \}$, $\tilde f_1$ on $I^n \times \{ 1 \}$ and $f$ is constant on $\partial I^n \times I$. The homotopy lifting property gives $\tilde H$ defining a homotopy between $\tilde f_0$ and $\tilde f_1$. 

The result then follows (modulo some noise in the low homotopy groups, which can be checked by hand) from \Cref{thm:les_rel}. 
\end{proof}
\begin{Exa}[Hopf fibrations]
	Let $\mathbb{F} = \mathbb{R},\mathbb{C}$ or $\mathbb{H}$ and fix an integer $d = 1,2$ or $4$, respectively. 

	Let 
	\[
\mathbb{F}^{n+1} = \begin{cases}
	\mathbb{R}^{n+1} & \mathbb{F} = \mathbb{R} \\
	\mathbb{C}^{n+1} \cong \mathbb{R}^{2(n+1)} & \mathbb{F} = \mathbb{C} \\
	\mathbb{H}^{n+1} \cong \mathbb{R}^{4(n+1)} & \mathbb{F} = \mathbb{H}.
\end{cases}
\]

In other words, $\mathbb{F}^{n+1} \cong \mathbb{R}^{d(n+1)}$. We define the $d(n+1)-1$ dimensional sphere inside $\mathbb{F}^{n+1}$:
\[
S^{d(n+1)-1} = \{ (u_0,\ldots,u_n) \mid u_i \in \mathbb{F}, \sum_{k=0}^n |u_k|^2 = 1 \}.
\]
We define the $\mathbb{F}$-projective space by
\[
\mathbb{F}P^n \coloneqq \mathbb{F}^{n+1} \setminus \{ 0 \} / \sim
\]
where $(u_0,\ldots,u_n) \simeq (v_0,\ldots,v_n)$ if and only if there exists $\lambda \in \mathbb{F} \setminus \{ 0 \}$ such that $v_i = \lambda u_i$ for $i = 0,\ldots,n$. 

Now we have a map $\phi \colon S^{d(n+1)-1} \to \mathbb{F}P^n$ that sends $(u_0,\ldots,u_n)$ to its equivalence class $[u_0,\ldots,u_n]$. Let $F = \phi^{-1}[1,\ldots,0] = \{ (\lambda, 0,\ldots,0) \mid \lambda \in \mathbb{F},|\lambda| = 1 \} \cong S^{d-1}$. 

We will see later in the course that $S^{d-1} \to S^{d(n+1)-1} \to \mathbb{R}P^n$ is a fibration. Explicitly, the fibrations are
\[
\begin{split}
S^0 &\to S^n \to \mathbb{R}P^n \\
S^1 &\to S^{2n+1} \to \mathbb{C}P^n \\
S^3 &\to S^{4n+3} \to \mathbb{H}P^n.
\end{split}
\]
The case $n = 1$ is of interest, as then projective spaces are just spheres, and we obtain the following Hopf fibrations 
\[
\begin{split}
S^0 &\to S^1 \to S^1 \\
S^1 &\to S^{3} \to S^2\\
S^3 &\to S^{7} \to S^4.
\end{split}
\]
There is also a fibration $S^7 \to S^{15} \to S^8$. It is a difficult theorem of Adams that these are the only fibrations between spheres. 
\end{Exa}
\end{document}